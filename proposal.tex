

\documentclass[conference]{IEEEtran}




% correct bad hyphenation here
\hyphenation{op-tical net-works semi-conduc-tor}


\begin{document}

\title{Assignment Helper\\ Empowering Your Programming Skill \\ Power OVERWHELMING co.}

% author names and affiliations
% use a multiple column layout for up to three different
% affiliations
% \author{\IEEEauthorblockN{Jae-gook Kim}
% \IEEEauthorblockA{Department of Information System\\
% Hanyang University, Seoul,  Korea\\
% Email: claretta@hanyang.ac.kr}
% \and
% \IEEEauthorblockN{Kyung-min Kim}
% \IEEEauthorblockA{Department of Information System\\
% Hanyang University, Seoul,  Korea\\
% Email: kimkmhk@hanyang.ac.kr}
% \IEEEauthorblockN{Ji-hoon Lee}
% \IEEEauthorblockA{Department of Finance,\\ Business School,\\
% Hanyang University, Seoul,  Korea\\
% Email: 1equal2@hanyang.ac.kr}
% \and
% \IEEEauthorblockN{Kyo-ho Lee}
% \IEEEauthorblockA{Department of Information System\\
% Hanyang University, Seoul,  Korea\\
% Email: 1equal2@hanyang.ac.kr}}


% conference papers do not typically use \thanks and this command
% is locked out in conference mode. If really needed, such as for
% the acknowledgment of grants, issue a \IEEEoverridecommandlockouts
% after \documentclass

% for over three affiliations, or if they all won't fit within the width
% of the page, use this alternative format:
% 
\author{\IEEEauthorblockN{Ji-hoon Lee\IEEEauthorrefmark{1},
Jae-gook Kim\IEEEauthorrefmark{2},
Kyung-min Kim\IEEEauthorrefmark{3}, and
Kyo-ho Lee\IEEEauthorrefmark{4} }
\IEEEauthorblockA{\IEEEauthorrefmark{2}\IEEEauthorrefmark{3}\IEEEauthorrefmark{4}Department of Information System}
\IEEEauthorblockA{\IEEEauthorrefmark{1}Department of Finance, Business School}
\IEEEauthorblockA{Hanyang University, Seoul, Korea}
\IEEEauthorblockA{Email: \IEEEauthorrefmark{1}starypoc@hanyang.ac.kr,  \IEEEauthorrefmark{2}claretta@hanyang.ac.kr, \IEEEauthorrefmark{3}kimkmhk@hanyang.ac.kr, \IEEEauthorrefmark{4}1equal2@hanyang.ac.kr}}



\maketitle

\begin{abstract}
Assignment Helper is a program to help people who have difficulties with their programming language assignments. Assignment Helper will able to help the assignment by crawling the data from webs such as programmer forums and then find similar questions and related codes. It will also have a verification function and will check the codes by using CRC or other compilers. It shall check it by testing the input, output and the expectation all together.
\end{abstract}

\IEEEpeerreviewmaketitle

\begin{keyword}
assignment, helper
\end{keyword}


\section{Introduction}
% no \IEEEPARstart
This demo file is intended to serve as a ``starter file''
for IEEE conference papers produced under \LaTeX\ using
IEEEtran.cls version 1.8b and later.
% You must have at least 2 lines in the paragraph with the drop letter
% (should never be an issue)
I wish you the best of success.

\hfill mds
 
\hfill August 26, 2015

\subsection{Subsection Heading Here}
Subsection text here.

% \begin{table}[!t]
% % increase table row spacing, adjust to taste
% \renewcommand{\arraystretch}{1.3}
% if using array.sty, it might be a good idea to tweak the value of
% % \extrarowheight as needed to properly center the text within the cells
% \caption{An Example of a Table}
% \label{table_example}
% \centering
% % Some packages, such as MDW tools, offer better commands for making tables
% % than the plain LaTeX2e tabular which is used here.
% \begin{tabular}{|c||c|}
% \hline
% One & Two\\
% \hline
% Three & Four\\
% \hline
% \end{tabular}
% \end{table}

\subsubsection{Subsubsection Heading Here}
Subsubsection text here.


\begin{thebibliography}{1}

\bibitem{IEEEhowto:kopka}
H.~Kopka and P.~W. Daly, \emph{A Guide to \LaTeX}, 3rd~ed.\hskip 1em plus
  0.5em minus 0.4em\relax Harlow, England: Addison-Wesley, 1999.

\end{thebibliography}




% that's all folks
\end{document}


